\chapter{Zahlentheoretische Funktionen}

\begin{definition}
    Eine \emph{zahlentheoretische Funktion} ist eine Abbildung $a : \N \to \C$. Einer solchen Funktion ordnet man die (formale) \emph{Dirichletsche Reihe} zu:
    $$ A(s) \coloneqq \sum_{n=1}^\infty \frac{a(n)}{n^s}, \quad s \in \C. $$
\end{definition}

\begin{example}
    Beispiele für zahlentheoretische Funktionen sind:
    \begin{align*}
        d(n) &\coloneqq \# \{ k : k \mid n \} \\
        \varphi(n) &\coloneqq \# \{ 1 \leq a \leq n \mid \ggT(a, n) = 1 \} \\
        \mu(n) &\coloneqq \left\{ \begin{array}{ll}
            1, & n = 1, \\
            (-1)^r, & n = p_1 \cdot p_2 \cdots p_r, \textrm{ wobei $p_j$ verschiende Primzahlen}, \\
            0, & \textrm{sonst}.
        \end{array} \right. \\
        \Lambda(n) &\coloneqq \left\{ \begin{array}{ll}
            \log p, & n = p^k, p \in \P, k \geq 1, \\
            0, & \textrm{sonst}.
        \end{array} \right.
    \end{align*}
\end{example}

\begin{definition}
    Für zahlentheoretische Funktionen $a, b$ definieren eine Summe
    $$ c(n) \coloneqq (a+b)(n) \coloneqq a(n) + b(n) \quad \leftrightarrow \quad C(s) = A(s) + B(s) $$
    und die \emph{Dirichlet-Faltung}:
    $$ c(n) \coloneqq (a \ast b)(n) = \sum_{d \mid n} a(d) b(n/d) = \sum_{d_1 d_2 = n} a(d_1) b(d_2) $$
    $$ \leftrightarrow \quad C(s) = \sum_{n=1}^\infty \frac{\sum_{d_1 d_2 = n} a(d_1) b(d_2)}{n^s} = \sum_{d_1 \geq 1} \frac{a(d_1)}{n^s} \cdot \sum_{d_2 \geq 1} \frac{b(d_2)}{n^s} = A(s) \cdot B(s). $$
\end{definition}

\begin{example}
    Die Dirchlet-Faltung besitzt ein neutrales Element:
    $$ I(n) = \left\{ \begin{array}{ll}
        1 & n = 1, \\
        0 & \textrm{sonst},
    \end{array} \right.
    \quad \sum_{n=1}^\infty \frac{I(n)}{n^s} = 1,
    \quad a \ast I = I \ast a = a $$
\end{example}

\begin{lemma}    
    Sei $a$ eine zahlentheoretische Funktion. Dann besitzt $a$ ein (bezüglich $\ast$) inverses Element $a^{-1}$ genau dann wenn $a(1) \neq 0$.
\end{lemma}

\begin{proof}{\ }
    \begin{itemize}
        \item[``$\Rightarrow$''] $ (a \ast a^{-1})(1) = a(1) \cdot a^{-1}(1) = 1 = I(1) \implies a(1) \neq 0 $
        \item[``$\Leftarrow$''] $ a^{-1}(1) \coloneqq \frac{1}{a(1)}, \quad a^{-1}(n) \coloneqq -\frac{1}{a(1)} \sum_{d \mid n, d < n} a(\frac{n}{d}) a^{-1}(d), n > 1 $
    \end{itemize}
\end{proof}

\begin{remark}
    Besitzt $a$ ein Inverses, so gilt für die Dirichletsche Reihe:
    $$ \sum_{n=1}^\infty \frac{a^{-1}(n)}{n^s} = \frac{1}{A(s)} $$
\end{remark}

\begin{definition}
    Sei $a$ eine zahlentheoretische Funktion $a$ mit $a \neq 0$.
    \begin{itemize}
        \item $a$ heißt \emph{multiplikativ}, falls $a \neq 0$ und aus $\ggT(m, n) = 1$ folgt $a(mn) = a(m) a(n)$.
        \item $a$ heißt \emph{vollständig multiplikativ}, falls $a \neq 0$ und stets $a(mn) = a(m) a(n)$.
    \end{itemize}
\end{definition}

\begin{remark}{\ }
    \begin{itemize}
        \item Klarerweise folgt aus vollständig multiplikativ auch multiplikativ.
        \item Ist $a$ multiplikativ so ist $a(1) = 1$, wegen $a(1) = a(1) a(1)$ und $a(n) a(1) = a(n)$.
        \item Ist $a$ multiplikativ, so legt $a(p^k), p \in \P, k \geq 1$ die Funktion bereits fest.
        \item Ist $a$ vollständig multiplikativ, so legt $a(p), p \in \P$ die Funktion bereits fest.
    \end{itemize}
\end{remark}

\begin{example}
    Die Funktion $J(n) = 1$ ist vollständig multiplikativ und die entsprechende Dirichletsche Reihe ist die \emph{Riemannsche Zeta-Funktion}:
    $$ \sum_{n=1}^\infty \frac{J(n)}{n^s} = \sum_{n=1}^\infty \frac{1}{n^s} \eqqcolon \zeta(s). $$
\end{example}

\begin{example}
    Die \emph{Möbius-Funktion}
    $$ \mu(n) \coloneqq \left\{ \begin{array}{ll}
        1, & n = 1, \\
        (-1)^r, & n = p_1 \cdot p_2 \cdots p_r, \textrm{ wobei $p_j$ verschiende Primzahlen}, \\
        0, & \textrm{sonst}.
    \end{array} \right. \\ $$
    ist multiplikativ und es gilt $(\mu \ast J)(1) = 1$. Ist $n = p_1^{k_1} \cdots p_r^{k_r}$ mit verschiedenen Primzahlen $p_j$, so folgt
    $$ (\mu \ast J)(n) = \sum_{d \mid n} \mu(d) = \mu(1) + \mu(p_1) + ... + \mu(p_r) + \mu(p_1 p_2) + ... + \mu(p_1 ... p_r) = (1 - 1)^r = 0, $$
    demnach ist $\mu = J^{-1}$ und damit
    $$ \sum_{n=1}^\infty \frac{\mu(n)}{n^s} = \frac{1}{\zeta(s)}. $$
\end{example}

\begin{example}
    Für die \emph{Von Mangoldtsche-Funktion} gilt $(\Lambda \ast J)(1) = 1$. Ist $n = p_1^{k_1} \cdots p_r^{k_r}$ mit verschiedenen Primzahlen $p_j$, so folgt
    $$ (\Lambda \ast J)(n) = \sum_{d \mid n} \Lambda(d) = k_1 \log p_1 + k_2 \log p_2 + ... k_r \log p_r = \log n, $$
    demnach ist $\Lambda = \log^{-1}$. Wegen $(n^{-s})' = -n^{-s} \log n$ erhalten wir weiters
    $$ \sum_{n=1}^\infty \frac{\log n}{n^s} = -\zeta'(s) \quad \textrm{und damit} \quad \sum_{n=1}^\infty \frac{\Lambda(n)}{n^s} = -\frac{\zeta'(s)}{\zeta(s)}. $$
\end{example}

\begin{example}
    Für die \emph{Eulersche Phi-Funktion} gilt
    $$ (\varphi \ast J)(n) = \sum_{d \mid n} \varphi(d) = n. $$
    Demnach erhalten wir
    $$ \sum_{n=1}^\infty \frac{n}{n^s} = \zeta(s-1) \quad \textrm{und} \quad \sum_{n=1}^\infty \frac{\varphi(n)}{n^s} = \frac{\zeta(s-1)}{\zeta(s)}. $$
\end{example}

\begin{theorem}
    Sei $a$ multiplikativ, dann ist
    $$ \sum_{n=1}^\infty \frac{a(n)}{n^s} = \prod_{p \in \P} \left( 1 + \frac{a(p)}{p^s} + \frac{a(p^2)}{p^{2s}} + ... \right). $$
    Ist $a$ vollständig multiplikativ, dann ist
    $$ \sum_{n=1}^\infty \frac{a(n)}{n^s} = \prod_{p \in \P} \frac{1}{1 - \frac{a(p)}{p^s}}. $$
\end{theorem}