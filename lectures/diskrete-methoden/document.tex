\documentclass[letterpaper, 12pt]{article}

\usepackage[ngerman]{babel}
\usepackage[dvipsnames]{xcolor}
\usepackage{
  amsmath, amsthm, amssymb, mathtools, dsfont, units,       
  graphicx, wrapfig, subfig, float,                         
  listings, color, inconsolata, pythonhighlight,            
  fancyhdr, hyperref, enumerate, enumitem, framed
}
\usepackage{hyperref} 
\usepackage{newpxtext, newpxmath, inconsolata}
\usepackage[left=1.35in, right=1.35in, top=1.0in, bottom=.9in, headsep=.2in, footskip=0.35in]{geometry}
\usepackage[bottom]{footmisc}
\usepackage{anyfontsize}


\renewcommand{\baselinestretch}{1.2}
\setlength{\parskip}{1.3mm}
\setlength{\headheight}{15pt}
\allowdisplaybreaks

\usepackage[font={it,footnotesize}]{caption}
\hypersetup{colorlinks=true, linkcolor=RoyalBlue, citecolor=RedOrange, urlcolor=ForestGreen}
\usepackage{titlesec}
\titleformat{\section}{\large\bfseries}{{\fontsize{19}{19}\selectfont\textreferencemark}\;\; }{0em}{}
\titleformat{\subsection}{\normalsize\bfseries\selectfont}{\thesubsection\;\;\;}{0em}{}
\setlist[itemize]{wide=0pt, leftmargin=16pt, labelwidth=10pt, align=left}



\usepackage[subfigure]{tocloft}
\setlength{\cftbeforesecskip}{.9ex}
\cftsetindents{section}{0em}{0em}
\renewcommand{\cfttoctitlefont}{\large\bfseries}

\makeatletter
\renewcommand{\cftsecpresnum}{\begin{lrbox}{\@tempboxa}}
\renewcommand{\cftsecaftersnum}{\end{lrbox}}
\makeatother

\graphicspath{{Images/}{../Images/}}



\newcommand\numberthis{\addtocounter{equation}{1}\tag{\theequation}}
\let\b\mathbf
\let\bg\boldsymbol
\let\mc\mathcal

\usepackage[mathscr]{euscript}

\newcommand{\R}{\mathbb{R}}
\newcommand{\C}{\mathbb{C}}
\newcommand{\D}{\mathbb{D}}
\newcommand{\K}{\mathbb{K}}
\newcommand{\Q}{\mathbb{Q}}
\newcommand{\Z}{\mathbb{Z}}
\newcommand{\N}{\mathbb{N}}
\newcommand{\F}{\mathcal{F}}
\newcommand{\El}{\mathcal{L}}
\newcommand{\1}{\mathds{1}}
\newcommand{\diff}{\mathop{}\!\mathrm{d}}
\newcommand{\clspan}{\mathrm{clspan}}
\newcommand{\mapping}[3]{
  \ensuremath{
    #1 : \left\{
    \begin{array}{@{\;}c@{\;}c@{\;}l@{}}
      #2 \\
      #3
    \end{array}
    \right.
  }
}


\DeclareMathOperator*{\argmax}{argmax}
\DeclareMathOperator*{\argmin}{argmin}
\DeclareMathOperator*{\spann}{Span}		
\DeclareMathOperator*{\bias}{Bias}		
\DeclareMathOperator*{\ran}{ran}			
\DeclareMathOperator*{\dv}{d\!}
\DeclareMathOperator*{\diag}{diag}		
\DeclareMathOperator*{\trace}{trace}	
\DeclareMathOperator*{\supp}{supp}		

\theoremstyle{definition}
\newtheorem{theorem}{Satz}[section]
\newtheorem{proposition}[theorem]{Proposition}
\newtheorem{lemma}[theorem]{Lemma}
\newtheorem{corollary}[theorem]{Korollar}
\newtheorem{definition}[theorem]{Definition}
\newtheorem{example}[theorem]{Beispiel}
\newtheorem{remark}[theorem]{Bemerkung}



\pagestyle{fancy}
\renewcommand{\footrulewidth}{0pt}
\renewcommand{\headrulewidth}{0pt}

\renewcommand{\sectionmark}[1]{\markright{#1} }
\fancyhead[R]{\small\textit{\nouppercase{\rightmark}}}

\fancyhead[L]{\hyperref[sec:contents]{\small Diskrete Methoden}}



\begin{document}



\title{Diskrete Methoden \\[0.3em]
\normalsize nach der 2025W Vorlesung von Alois Panholzer}



\author{\normalsize Ian Hornik}
\date{\vspace{-0.8em}\normalsize\today}



\maketitle
\tableofcontents\label{sec:contents}



\section{Abzählprobleme}

Sei $\mathcal{A}$ eine Menge und $w : \mathcal{A} \to \N$ eine Gewichtsfunktion mit der Restriktion $w^{-1}(n) = \vert \{ \sigma \in \mathcal{A} : w(\sigma) = n \} \vert < \infty$ für alle $n \in \N$.

\begin{example}
    Wir betrachten die Familie der Binärbäume, also ein Baum mit der Eigenschaft dass jeder Knoten entweder $0$ oder $2$ Kinder hat. Erstere Knoten werden auch \emph{externe Knoten} (oder \emph{Blatt}) genannt, letzere auch \emph{interne Knoten}. Als Gewichtsfunktion $w(t)$ wollen wir die Anzahl der internen Knoten von $t$ setzen, also die ``Größe'' von $t$.

    Wir stellen uns die Frage wieviele Binärbäume es zu einer fixen Größe gibt. Setzen wir $b_n \coloneqq \vert w^{-1}(n) \vert$, so erkannt man leicht, dass 
    $$ b_0 = 1, b_1 = 1, b_2 = 2, b_3 = 5. $$

    Man kann folgern
    $$ b_{n+1} = \sum_{k=0}^n b_k b_{n-k}, \quad n > 0 $$
    mit $b_0 = 1$.
\end{example}

\begin{example}
    Wir betrachten reguläre Ausdrücke und zulässige Klammerungen. Sei $\circ$ eine binäre Operation, welche nicht assoziativ und nicht kommutativ ist. Wir betrachten einen Ausdruck der Form
    $$ x_1 \circ x_2 \circ \dots \circ x_{n+1} $$
    mit Variablen $x_1, \dots, x_{n+1}$. Es stellt sich die Frage auf wieviele (verschiedene) Arten dieser Ausdruck durch Klammerung ausgewertet werden kann. Diese Klammerungen entsprechen auf eindeutiger Weise Binärbaumen, bei denen die Wurzeln gerade die Variablen sind.

    Ist $\pi \in S_{n+1}$, so wollen wir wissen auf wieviele Arten man Ausdrücke der Form $x_{\pi(1)} \circ \dots \circ x_{\pi(n+1)}$ zulässig Klammern kann. Bezeichnen wir diese Anzahl mit $a_n$, so folgt
    $$ a_n = (n+1)! b_n. $$

    Es stellt sich die Frage auf wie viele Arten wir aus zulässigen Klammerungen von $x_1, \dots, x_{n+1}$ ein $x_{n+1}$ einfügen können um auf eine zulässige Klammerung von $x_1, \dots, x_{n+2}$ zu erhalten.
\end{example}




\end{document}
